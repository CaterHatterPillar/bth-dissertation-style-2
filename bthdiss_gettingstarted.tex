%% BTH Dissertation Style 2 (see https://github.com/CaterHatterPillar/bth-dissertation-style-2)
%% Blekinge Institute of Technology alternative dissertation/thesis template. See https://github.com/CaterHatterPillar/bth-dissertation-style for a style resembling the official, close-source, dissertation template commonly used in BTH computer science theses.
%%
%% Based on the ILLC Dissertation Style by the Institute for Logic, Language and Computation, University of Amsterdam (used with permission).
%% Licensed under the BSD 3-Clause License. See adjoined LICENSE for more information.
%% Currently maintained by Eric Nilsson.
%%
%% Version: June 29, 2014.
%%

\chapter{Getting started}
This file describes the BTH Dissertation Style 2 package for
typesetting dissertations in \LaTeX .
The document describes which files are needed, and how they should be adopted
for your dissertation.
It also serves as an example of using these files, and as a template
for your own dissertation.

The BTH Dissertation Style file will change the
layout of your dissertation to a recommended Dissertation Style.
Amongst other things, it defines a standard layout for the cover and spine of your dissertation.
Furthermore, it redefines the layout of \verb|\chapter|, page heads, and theorem-like environments, and provides predefined theorem-like environments and commands for special sections such as \verb|\acknowledgements|.
Note that the style does not serve as an \textit{official} dissertation template (yet) for theses submitted to Blekinge Institute of Technology.
However, there are \href{https://github.com/CaterHatterPillar/bth-dissertation-style}{open-source alternatives} (see https://github.com/CaterHatterPillar/bth-dissertation-style) for such styles.

If you are already familiar with the standard {\tt book.cls} provided with
\LaTeX 2$\epsilon$, then the BTH Dissertation Style file should not give you
any difficulties: you may use all {\tt book} style commands to prepare 
your dissertation.
For a description of the commands available in the \LaTeX 2$\epsilon$\ 
{\tt book} style we refer you to the {\em \LaTeX{} User's Guide \& Reference
Manual\/} by Leslie Lamport (1986, 1994), Addison-Wesley Publishing
Company, Reading, Mass.

\section{How to proceed}
The BTH Dissertation Style 2 package contains the following main files:

\begin{description} 
	\item[{\tt bthdiss.cls}:] the BTH Dissertation Style for use with \LaTeX 2$\epsilon$
	\item[{\tt bthdiss\_logo.eps}:] the BTH logo; input by {\tt bthdiss\_front.tex} and {\tt bthdiss\_spine}.
	\item[{\tt bthdiss.tex}:] the main latex file for this document; acting as a reference guideline for the BTH Dissertation Style.
	\item[{\tt bthdiss\_front.tex}:] file describing the BTH Dissertation Style front matter.
	\item[{\tt bthdiss\_spine.tex}:] file for preparing the text for the spine of your dissertation (if applicable).
\end{description}

You should make sure that \LaTeX\ is able to find the files {\tt bthdiss.cls} and {\tt bthdiss\_logo.eps} when you typeset your document with the BTH Dissertation Style; the simple way to achieve this being to put all files in the BTH Dissertation Style package in the directory where your source files reside.

\section{Invoking the ILLC Dissertation Style}
The BTH Dissertation Style 2 is invoked by replacing ``book'' by ``bthdiss'' in the first line of your document. You should also \verb|\include| 
a {\em personalized\/} version of the file {\tt bthdiss\_front.tex} after the \verb|\begin{document}| declaration.

\begin{verbatim}
\documentclass{bthdiss}

\begin{document}
\pagestyle{plain}
\pagenumbering{roman}

%% Front matter:
%% BTH Dissertation Style 2 (see https://github.com/CaterHatterPillar/bth-dissertation-style-2)
%% Blekinge Institute of Technology alternative dissertation/thesis template. See https://github.com/CaterHatterPillar/bth-dissertation-style for a style resembling the official, close-source, dissertation template commonly used in BTH computer science theses.
%%
%% Based on the ILLC Dissertation Style by the Institute for Logic, Language and Computation, University of Amsterdam (used with permission).
%% Licensed under the BSD 3-Clause License. See adjoined LICENSE for more information.
%% Currently maintained by Eric Nilsson.
%%
%% Author: Maarten de Rijke
%% Current maintainer: Marco Vervoort
%% Editor: Eric Nilsson (used with permission)
%%
%% Version: June 29, 2014.
%%

\newcommand{\printtitle}{
	\begin{Huge}
		% TODO: add thesis title:
		\textbf{The BTH Dissertation Style\\[0.8cm] 2}
	\end{Huge}
}

% Front page:
{
	\setlength{\parindent}{0pt}
	\pagestyle{empty}
	\begin{titlepage}
		\par\vskip 2cm

		% Title and Author(s):
		\begin{center}
			\printtitle
		
			\vfill
		
			\begin{LARGE}
				% TODO: add thesis author(s):
				\textbf{Firstname Lastname}
			\end{LARGE}

			\vskip 2cm % Distance from margin.
		\end{center}
	\end{titlepage}

	% TODO: correct thesis layout if no complementing empty page is desired:
	% Skip a page to start on a right page again.
	% If you're printing single-sided, simply delete
	% the following line.
	%
	\mbox{}\newpage
	\setcounter{page}{1}

	%% Additional title page; the `franse pagina':
	\par\vskip 2cm
	\begin{center}
		\printtitle
	\end{center}

	%% Institution info page:
	\clearpage
	\par\vskip 2cm
	\begin{center}
		% TODO: add thesis number (this is commonly given to you only after your thesis has been graded):
		Thesis no: \textbf{DEPT}-20\textbf{XX}-\textbf{XX}
		
		\par\vspace {2cm}
		
		\bthlogo{4cm}
		
		\par\vspace {2cm}

		% TODO: complement with the correct BTH institution department:
		For further information about publications, please contact:\\[2ex]
		Faculty of Computing\\
		Blekinge Institute of Technology\\
		Valhallavägen\\
		SE-371 79, Karlskrona

		\par\vspace{0.5cm} % Originally no vertical spacing.

		Internet: \href{http://www.bth.se}{www.bth.se}\\
		Phone: +46 455 38 50 00\\
		Fax: +46 455 38 50 57
	\end{center}

	\vfill

	% TODO: if your project is sponsored by an organization or corporation (or similar), add appropriate attribution here:
	The investigations were supported by the Miskatonic University Research Foundation, which is subsidized by the Massachusetts Organization for Scientific Research.

	\par\vspace {2cm}

	% If you want to add CIP data (a summary of all the data used in
	% library catalogs in a standard format), uncomment the following
	% three lines and add the CIP data in between
	%
	%\noindent%
	%{\tt CIP gegevens}                                 % PERSONALIZE
	%\\[4ex]                                            %PERSONALIZE

	% TODO: if your thesis requires any additional mention of copyright, patents, ISBN, cover design, or similar, input said information here:
	Copyright \copyright\ 20\textbf{XX} by Firstname Lastname\\[2ex]
	Cover design by Firstname Lastname.\\ % If your cover was designed by someone else.
	% Maybe some additional info on the production of the dissertation in case of industrial collaboration.
	Printed and bound by your printer.\\[2ex] %Don't forget your printing shop
	ISBN: 90--\textbf{XXXX}--\textbf{XXX}--\textbf{X} % ISBN number: ask your faculty library how to obtain one (if applicable).

	%% Third titlepage:
	\clearpage
	\par\vskip 2cm

	\begin{center}
		\printtitle

		\par\vspace {6cm}
		\begin{large}
			\textsc{Academic Thesis}
		\end{large}
		
		\par\vspace {1cm}

		% TODO: complement with the degree you're aspiring to attain, and opposition information (if applicable).
		\begin{large}
			to obtain the degree of doctor at Blekinge Institute of Technology\\
			on the authority of Rector Prof. Firstname Lastname,\\
			opposed by the Doctorate Committee,\\
			and defended at the University in public\\
			on Monday, January 1st, 2014, at 12:00 am
		\end{large}

		\par\vspace{1cm}
		\begin{large}
			by
		\end{large}

		% TODO: complement with your _full_ name:
		\par\vspace{1cm}
		\begin{Large}
			Title Firstname Middlename Lastname
		\end{Large}

		% TODO: add your place of birth:
		\par\vspace{0.5cm}
		\begin{large}
			born in Arkham, Massachusetts, the United States of America.
		\end{large}
	\end{center}

	% Additional info page, advisors, corporate supervisors, etc:
	\clearpage

	% TODO: add contact information of author(s):
	\textbf{Contact Information:}\\
	Author(s):\\
	Firstname Lastname \\
	E-mail: \href{mailto:acroyy@student.bth.se}{acroyy@student.bth.se}

	\bigskip

	% TODO: add name of any external advisors which may apply:
	External advisor: \\
	Title Firstname Lastname\\
	Example Corporation AB

	\bigskip

	% TODO: add name of your university advisor:
	University advisor:\\
	Prof. Firstname Lastname,\\
	Research Dean of Faculty\\ % extra title, if applicable.
	Faculty of Computing

	\vfill

	\begin{tabular}{p{0.5\linewidth}lcl}
		Faculty of Computing				& Internet	& : & \href{http://www.bth.se}{www.bth.se}\\
		Blekinge Institute of Technology	& Phone		& : & +46 455 38 50 00\\
		SE--371 79 Karlskrona, Sweden		& Fax		& : & +46 455 38 50 57\\
	\end{tabular}

	% In order to offset the department from page bottom:
	\bigskip \bigskip % This ought be replaced by simply setting the page bottom margin similar to that of the top margin.

	\clearpage
}

%% BTH Dissertation Style 2
%% Blekinge Institute of Technology alternative dissertation/thesis template. See https://github.com/CaterHatterPillar/bth-dissertation-style for a style resembling the official, close-source, dissertation template commonly used in BTH computer science theses.
%%
%% Based on the ILLC Dissertation Style by the Institute for Logic, Language and Computation, University of Amsterdam (used with permission).
%% Licensed under the BSD 3-Clause License.
%% See adjoined LICENSE for more information.

\thispagestyle{plain}
\mbox{}

\vspace{2in}

% TODO: inser optional dedication.
\begin{center}
	{\em
		To my Sun\\
		and Stars
	}
\end{center}

\tableofcontents
%% BTH Dissertation Style 2
%% Blekinge Institute of Technology alternative dissertation/thesis template. See https://github.com/CaterHatterPillar/bth-dissertation-style for a style resembling the official, close-source, dissertation template commonly used in BTH computer science theses.
%%
%% Based on the ILLC Dissertation Style by the Institute for Logic, Language and Computation, University of Amsterdam (used with permission).
%% Licensed under the BSD 3-Clause License.
%% See adjoined LICENSE for more information.

\acknowledgments

I would like to express my most respectful gratitude to Azathoth, Ghatanothoa, and Shub-Niggurath; without whom I would never have completed the task given.
Furthmore, I must acknowledge the contributions of Yog-Sothoth, Nyarlathotep, and Yig, all three of whom have contributed greatly to the conclusions presented in this thesis.\\
Cthulhu showed me the way, as he hath done unto many before me.

\vspace{0.5cm}

{\em \noindent Fhtagn, fhtagn!}\\[2ex]		
Arkham\hfill Firstname Initial Lastname\\
\textbf{Month}, 20\textbf{XX}.


%% Body matter:
\cleardoublepage
\pagestyle{headings}
\pagenumbering{arabic}
  
    <your dissertation>

%% Back matter:
%% BTH Dissertation Style 2
%% Blekinge Institute of Technology alternative dissertation/thesis template. See https://github.com/CaterHatterPillar/bth-dissertation-style for a style resembling the official, close-source, dissertation template commonly used in BTH computer science theses.
%%
%% Based on the ILLC Dissertation Style by the Institute for Logic, Language and Computation, University of Amsterdam (used with permission).
%% Licensed under the BSD 3-Clause License. See adjoined LICENSE for more information.
%% Currently maintained by Eric Nilsson.
%%
%% Version: June 29, 2014.
%%

\begin{thebibliography}{XX}
\bibitem{Comment}
Preferably, bibliographic references should always be included in your dissertation.
It is specified by:
\begin{verbatim}
  \begin{thebibliography}{XX}
    <your list of \bibitems>
  \end{thebibliography}
\end{verbatim}
\bibitem{Lamport}
L. Lamport. {\em \LaTeX{} User's Guide \& Reference
Manual\/}, Addison-Wesley Publishing Company, Reading, Mass. 1986, 1994.
\end{thebibliography}

%% BTH Dissertation Style 2
%% Blekinge Institute of Technology alternative dissertation/thesis template. See https://github.com/CaterHatterPillar/bth-dissertation-style for a style resembling the official, close-source, dissertation template commonly used in BTH computer science theses.
%%
%% Based on the ILLC Dissertation Style by the Institute for Logic, Language and Computation, University of Amsterdam (used with permission).
%% Licensed under the BSD 3-Clause License. See adjoined LICENSE for more information.
%% Currently maintained by Eric Nilsson.
%%
%% Version: June 29, 2014.
%%

\begin{theindex}
	By preference, your dissertation\linebreak
	should contain an index. Instructions
	on how to produce an index can be
	found on pages 77--79 of the
 	\LaTeX\ manual. You may specify
	an index as follows:\\[2ex]
	\verb|  \begin{theindex}|\\
	\verb|    <your list of entries>|\\
	\verb|  \end{theindex}|
\end{theindex}

%% BTH Dissertation Style 2 (see https://github.com/CaterHatterPillar/bth-dissertation-style-2)
%% Blekinge Institute of Technology alternative dissertation/thesis template. See https://github.com/CaterHatterPillar/bth-dissertation-style for a style resembling the official, close-source, dissertation template commonly used in BTH computer science theses.
%%
%% Based on the ILLC Dissertation Style by the Institute for Logic, Language and Computation, University of Amsterdam (used with permission).
%% Licensed under the BSD 3-Clause License. See adjoined LICENSE for more information.
%% Currently maintained by Eric Nilsson.
%%
%% Version: June 29, 2014.
%%

\begin{thesymbols}
This is an optional chapter containing a list of symbols that
you use. It is specified by:\\[2ex]
\verb|  \begin{thesymbols}|\\
\verb|    <your list of symbols>|\\
\verb|  \end{thesymbols}|
\end{thesymbols}

%% BTH Dissertation Style 2 (see https://github.com/CaterHatterPillar/bth-dissertation-style-2)
%% Blekinge Institute of Technology alternative dissertation/thesis template. See https://github.com/CaterHatterPillar/bth-dissertation-style for a style resembling the official, close-source, dissertation template commonly used in BTH computer science theses.
%%
%% Based on the ILLC Dissertation Style by the Institute for Logic, Language and Computation, University of Amsterdam (used with permission).
%% Licensed under the BSD 3-Clause License. See adjoined LICENSE for more information.
%% Currently maintained by Eric Nilsson.
%%
%% Version: June 29, 2014.
%%

\abstract
Preferably, an abstract of your dissertation should be included.
This chapter may be specified by:
\begin{verbatim}
  \abstract
    <your Abstract>
\end{verbatim}

%% BTH Dissertation Style 2 (see https://github.com/CaterHatterPillar/bth-dissertation-style-2)
%% Blekinge Institute of Technology alternative dissertation/thesis template. See https://github.com/CaterHatterPillar/bth-dissertation-style for a style resembling the official, close-source, dissertation template commonly used in BTH computer science theses.
%%
%% Based on the ILLC Dissertation Style by the Institute for Logic, Language and Computation, University of Amsterdam (used with permission).
%% Licensed under the BSD 3-Clause License. See adjoined LICENSE for more information.
%% Currently maintained by Eric Nilsson.
%%
%% Version: June 29, 2014.
%%

\curriculum
This is an optional chapter containing your Curriculum Vitae.
It is specified as follows:
\begin{verbatim}
  \curriculum
    <your CV>
\end{verbatim}



\end{document}
\end{verbatim}
%
%
If your thesis is already coded with \LaTeX{} you can easily
adapt it a posteriori to the BTH Dissertation Style 2.

If your document is coded with the BTH Dissertation Style 2, however, you may not be able to typeset it using the standard \LaTeX\ book style
without doing some minor recoding, as the BTH Dissertation Style defines some comands that are not provided by the standard \LaTeX\ book style.

Please refrain from using any \LaTeX{} or \TeX{} commands that affect the layout or formatting of your document (i.e.\ commands like \verb|\textheight|, \verb|\hoffset| etc.).
The BTH Dissertation Style 2 has been carefully designed to produce the right layout from your \LaTeX\ input.
There may nevertheless be exceptional occasions on which to use some of them.
If there is anything specific you would like to do and for which neither \LaTeX{} nor the BTH Dissertation Style 2 file 
provides a command, {\em please contact me\/} (email: EricNNilsson@gmail.com).

\section{Personalizing {\tt bthdiss\_front.tex}}
The file {\tt bthdiss\_front.tex} contains all information needed to produce the front matter of your dissertation.
You need to personalize {\tt bthdiss\_front.tex} by inserting your data at appropriate spots.
Additionally, those that print on single-sided printers will want to eliminate the empty page printed after the cover page.
All items that need to be personalized in the BTH Dissertation Template 2 may be found by grepping the string ``{\% TODO:}''.
The files {\tt bthdiss\_dedication} and {\tt bthdiss\_acknowledgements} contain the optional dedication and acknowledgements, and are \verb|\include|d
from the main file.
You can personalize the text in these files, or simply change the name of the \verb|\include|d files in the main file.

\section{Personalizing {\tt bthdiss\_spine.tex}}
The file {\tt bthdiss\_spine.tex} contains all information needed to produce the spine of your dissertation (if applicable.
You need to personalize the file {\tt bthdiss\_spine.tex} by inserting your data at appropriate spots.
