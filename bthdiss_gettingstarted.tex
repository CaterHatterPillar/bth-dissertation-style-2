%% BTH Dissertation Style 2
%% Blekinge Institute of Technology alternative dissertation/thesis template. See https://github.com/CaterHatterPillar/bth-dissertation-style for a style resembling the official, close-source, dissertation template commonly used in BTH computer science theses.
%%
%% Based on the ILLC Dissertation Style by the Institute for Logic, Language and Computation, University of Amsterdam (used with permission).
%% Licensed under the BSD 3-Clause License. See adjoined LICENSE for more information.
%% Currently maintained by Eric Nilsson.
%%
%% Version: June 29, 2014.
%%

\chapter{Getting started}
This file describes the BTH Dissertation Style 2 package for
typesetting dissertations in \LaTeX .
The document describes which files are needed, and how they should be adopted
for your dissertation.
It also serves as an example of using these files, and as a template
for your own dissertation.

The BTH Dissertation Style file will change the
layout of your dissertation to a recommended Dissertation Style.
Amongst other things, it defines a standard layout for the cover and spine of your dissertation.
Furthermore, it redefines the layout of \verb|\chapter|, page heads, and theorem-like environments, and provides predefined theorem-like environments and commands for special sections such as \verb|\acknowledgements|.
Note that the style does not serve as an \textit{official} dissertation template (yet) for theses submitted to Blekinge Institute of Technology.
However, there are \href{https://github.com/CaterHatterPillar/bth-dissertation-style}{open-source alternatives} (see https://github.com/CaterHatterPillar/bth-dissertation-style) for such styles.

If you are already familiar with the standard {\tt book.cls} provided with
\LaTeX 2$\epsilon$, then the BTH Dissertation Style file should not give you
any difficulties: you may use all {\tt book} style commands to prepare 
your dissertation.
For a description of the commands available in the \LaTeX 2$\epsilon$\ 
{\tt book} style we refer you to the {\em \LaTeX{} User's Guide \& Reference
Manual\/} by Leslie Lamport (1986, 1994), Addison-Wesley Publishing
Company, Reading, Mass.

\section{How to proceed}
The BTH Dissertation Style 2 package contains the following main files:

\begin{description} 
	\item[{\tt bthdiss.cls}:] the BTH Dissertation Style for use with \LaTeX 2$\epsilon$
	\item[{\tt bthdiss\_logo.eps}:] the BTH logo; input by {\tt bthdiss\_front.tex} and {\tt bthdiss\_spine}.
	\item[{\tt bthdiss.tex}:] the main latex file for this document; acting as a reference guideline for the BTH Dissertation Style.
	\item[{\tt bthdiss\_front.tex}:] file describing the BTH Dissertation Style front matter.
	\item[{\tt bthdiss\_spine.tex}:] file for preparing the text for the spine of your dissertation (if applicable).
\end{description}

You should make sure that \LaTeX\ is able to find the files {\tt bthdiss.cls} and {\tt bthdiss\_logo.eps} when you typeset your document with the BTH Dissertation Style; the simple way to achieve this being to put all files in the BTH Dissertation Style package in the directory where your source files reside.

\section{Invoking the ILLC Dissertation Style}
The BTH Dissertation Style 2 is invoked by replacing ``book'' by ``bthdiss'' in the first line of your document. You should also \verb|\include| 
a {\em personalized\/} version of the file {\tt bthdiss\_front.tex} after the \verb|\begin{document}| declaration.

\begin{verbatim}
\documentclass{bthdiss}

\begin{document}
\pagestyle{plain}
\pagenumbering{roman}

%% Front matter:
\include{bthdiss_front}
\include{bthdiss_dedication}
\tableofcontents
%% BTH Dissertation Style 2
%% Blekinge Institute of Technology alternative dissertation/thesis template. See https://github.com/CaterHatterPillar/bth-dissertation-style for a style resembling the official, close-source, dissertation template commonly used in BTH computer science theses.
%%
%% Based on the ILLC Dissertation Style by the Institute for Logic, Language and Computation, University of Amsterdam (used with permission).
%% Licensed under the BSD 3-Clause License.
%% See adjoined LICENSE for more information.

\acknowledgments

I would like to express my most respectful gratitude to Azathoth, Ghatanothoa, and Shub-Niggurath; without whom I would never have completed the task given.
Furthmore, I must acknowledge the contributions of Yog-Sothoth, Nyarlathotep, and Yig, all three of whom have contributed greatly to the conclusions presented in this thesis.\\
Cthulhu showed me the way, as he hath done unto many before me.

\vspace{0.5cm}

{\em \noindent Fhtagn, fhtagn!}\\[2ex]		
Arkham\hfill Firstname Initial Lastname\\
\textbf{Month}, 20\textbf{XX}.


%% Body matter:
\cleardoublepage
\pagestyle{headings}
\pagenumbering{arabic}
  
    <your dissertation>

%% Back matter:
\include{bthdiss_bib}
%% BTH Dissertation Style 2
%% Blekinge Institute of Technology alternative dissertation/thesis template. See https://github.com/CaterHatterPillar/bth-dissertation-style for a style resembling the official, close-source, dissertation template commonly used in BTH computer science theses.
%%
%% Based on the ILLC Dissertation Style by the Institute for Logic, Language and Computation, University of Amsterdam (used with permission).
%% Licensed under the BSD 3-Clause License. See adjoined LICENSE for more information.
%% Currently maintained by Eric Nilsson.
%%
%% Version: June 29, 2014.
%%

\begin{theindex}
	By preference, your dissertation\linebreak
	should contain an index. Instructions
	on how to produce an index can be
	found on pages 77--79 of the
 	\LaTeX\ manual. You may specify
	an index as follows:\\[2ex]
	\verb|  \begin{theindex}|\\
	\verb|    <your list of entries>|\\
	\verb|  \end{theindex}|
\end{theindex}

%% BTH Dissertation Style 2
%% Blekinge Institute of Technology alternative dissertation/thesis template. See https://github.com/CaterHatterPillar/bth-dissertation-style for a style resembling the official, close-source, dissertation template commonly used in BTH computer science theses.
%%
%% Based on the ILLC Dissertation Style by the Institute for Logic, Language and Computation, University of Amsterdam (used with permission).
%% Licensed under the BSD 3-Clause License. See adjoined LICENSE for more information.
%% Currently maintained by Eric Nilsson.
%%
%% Version: June 29, 2014.
%%

\begin{thesymbols}
This is an optional chapter containing a list of symbols that
you use. It is specified by:\\[2ex]
\verb|  \begin{thesymbols}|\\
\verb|    <your list of symbols>|\\
\verb|  \end{thesymbols}|
\end{thesymbols}

%% BTH Dissertation Style 2
%% Blekinge Institute of Technology alternative dissertation/thesis template. See https://github.com/CaterHatterPillar/bth-dissertation-style for a style resembling the official, close-source, dissertation template commonly used in BTH computer science theses.
%%
%% Based on the ILLC Dissertation Style by the Institute for Logic, Language and Computation, University of Amsterdam (used with permission).
%% Licensed under the BSD 3-Clause License.
%% See adjoined LICENSE for more information.

\abstract
According to both ILLC standards and UvA promotion regulations,
an abstract of your dissertation in English should always be included.
This chapter may be specified by:
\begin{verbatim}
  \abstract
    <your Abstract>
\end{verbatim}

%% BTH Dissertation Style 2
%% Blekinge Institute of Technology alternative dissertation/thesis template. See https://github.com/CaterHatterPillar/bth-dissertation-style for a style resembling the official, close-source, dissertation template commonly used in BTH computer science theses.
%%
%% Based on the ILLC Dissertation Style by the Institute for Logic, Language and Computation, University of Amsterdam (used with permission).
%% Licensed under the BSD 3-Clause License.
%% See adjoined LICENSE for more information.

\curriculum
This is an optional chapter containing your Curriculum Vitae.
It is specified as follows:
\begin{verbatim}
  \curriculum
    <your CV>
\end{verbatim}



\end{document}
\end{verbatim}
%
%
If your thesis is already coded with \LaTeX{} you can easily
adapt it a posteriori to the BTH Dissertation Style 2.

If your document is coded with the BTH Dissertation Style 2, however, you may not be able to typeset it using the standard \LaTeX\ book style
without doing some minor recoding, as the BTH Dissertation Style defines some comands that are not provided by the standard \LaTeX\ book style.

Please refrain from using any \LaTeX{} or \TeX{} commands that affect the layout or formatting of your document (i.e.\ commands like \verb|\textheight|, \verb|\hoffset| etc.).
The BTH Dissertation Style 2 has been carefully designed to produce the right layout from your \LaTeX\ input.
There may nevertheless be exceptional occasions on which to use some of them.
If there is anything specific you would like to do and for which neither \LaTeX{} nor the BTH Dissertation Style 2 file 
provides a command, {\em please contact me\/} (email: EricNNilsson@gmail.com).

\section{Personalizing {\tt bthdiss\_front.tex}}
The file {\tt bthdiss\_front.tex} contains all information needed to produce the front matter of your dissertation.
You need to personalize {\tt bthdiss\_front.tex} by inserting your data at appropriate spots.
Additionally, those that print on single-sided printers will want to eliminate the empty page printed after the cover page.
All items that need to be personalized in the BTH Dissertation Template 2 may be found by grepping the string ``{\% TODO:}''.
The files {\tt bthdiss\_dedication} and {\tt bthdiss\_acknowledgements} contain the optional dedication and acknowledgements, and are \verb|\include|d
from the main file.
You can personalize the text in these files, or simply change the name of the \verb|\include|d files in the main file.

\section{Personalizing {\tt bthdiss\_spine.tex}}
The file {\tt bthdiss\_spine.tex} contains all information needed to produce the spine of your dissertation (if applicable.
You need to personalize the file {\tt bthdiss\_spine.tex} by inserting your data at appropriate spots.
