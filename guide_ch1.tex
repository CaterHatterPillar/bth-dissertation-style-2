%% BTH Dissertation Style 2
%% Blekinge Institute of Technology alternative dissertation/thesis template. See https://github.com/CaterHatterPillar/bth-dissertation-style for a style resembling the official, close-source, dissertation template commonly used in BTH computer science theses.
%%
%% Based on the ILLC Dissertation Style by the Institute for Logic, Language and Computation, University of Amsterdam (used with permission).
%% Licensed under the BSD 3-Clause License.
%% See adjoined LICENSE for more information.

\chapter{Getting started}

This file describes the ILLC Dissertation Style package for
typesetting dissertations in \LaTeX\ according to ILLC standards.
It describes which files are needed, and how they should be adopted
for your dissertation.
It also serves as an example of using these files, and as a template
for your own dissertation.

The ILLC Dissertation Style file will change the
layout of your dissertation to the required ILLC Dissertation Style.
It defines a standard layout for the cover and spine of your dissertation,
and includes a list of previous publications in the ILLC Dissertation series
Furthermore, it redefines the layout of \verb|\chapter|, page heads,
and theorem-like environments,
and provides predefined theorem-like environments and
commands for special sections such as \verb|\acknowledgements|.

If you are already familiar with the standard {\tt book.cls} provided with
\LaTeX 2$\epsilon$, then the ILLC Dissertation Style file should not give you
any difficulties: you may use all {\tt book} style commands to prepare 
your dissertation.
For a description of the commands available in the \LaTeX 2$\epsilon$\ 
{\tt book} style we refer you to the {\em \LaTeX{} User's Guide \& Reference
Manual\/} by Leslie Lamport (1986, 1994), Addison-Wesley Publishing
Company, Reading, Mass.

For the sake of compatibility, this package contains an old version of the
ILLC Dissertation Style, for use with \LaTeX 2.09. However, this version
is no longer supported, and we kindly request you to use the
\LaTeX 2$\epsilon$ version if at all possible.

\section{How to proceed}
The complete ILLC Dissertation Style package contains the following files:
\begin{description} 
\item[{\tt illcdiss.cls}:] the ILLC Dissertation Style for
use with \LaTeX 2$\epsilon$
\item[{\tt illc\_diss.sty}:] the ILLC Dissertation Style for
use with \LaTeX 2.09
\item[{\tt id10.sty}, {\tt id11.sty}, {\tt id12.sty}, {\tt epsf.sty}:] 
  auxiliary files for the ILLC Dissertation Style style,
  for use with \LaTeX 2.09
\item[{\tt illcdissertations.tex}:] file containing data on previous
  ILLC Dissertations
\item[{\tt illclogo.eps}:] this is the ILLC logo;
  input by {\tt guide\_front.tex}
\item[{\tt illc\_no\_text\_logo.eps}:] the ILLC logo without text;
  input by {\tt guide\_spine.tex}
\item[{\tt illclogo.pdf}, {\tt illc\_no\_text\_logo.pdf}:] PDF versions of the logos, used by {\tt pdflatex} instead of the EPS versions
\item[{\tt guide.tex}:] the main latex file for this document
\item[{\tt guide\_front.tex}:] file describing the official 
  ILLC-Dissertation front matter
\item[{\tt guide\_XXX.tex}:] file containing the text of section XXX of this 
document
\item[{\tt guide\_spine.tex}:] file for preparing the text 
  for the spine of your dissertation
\end{description}
You should make sure that \LaTeX\ is able to find the files
{\tt illcdissertations.tex}, {\tt illcdiss.cls}, {\tt illclogo.eps} and
{\tt illc\_no\_text\_logo.eps} when you 
typeset your document with the ILLC Dissertation Style; one way
to achieve this is to put all files in the ILLC Dissertation Style
package in the directory (or folder) where your dissertation files
reside.

Note that the {\tt illcdissertations.tex}
file in the archive is automatically updated for any new dissertations:
please download the most recent version 
before sending your dissertation to the printers.

\section{Invoking the ILLC Dissertation Style}
The ILLC Dissertation Style is invoked by replacing ``book'' by ``illcdiss''
in the first line of your document. You should also \verb|\include| 
a {\em personalized\/} version of the file {\tt guide\_front.tex} 
after the \verb|\begin{document}| declaration. 
You also need to \verb|\include| the file
\verb|illcdissertations.tex| after the last page of your dissertation:

\begin{verbatim}
\documentclass{illcdiss}

\begin{document}
\pagestyle{plain}
\pagenumbering{roman}

%%  \include the `front matter'

%% This is the standard `front matter' to be used with the illcdiss.cls
%% Latex2e document class or the illc_diss.sty Latex2.09 style file
%%
%% Author: Maarten de Rijke
%% Current maintainer: Marco Vervoort
%%
%% Version: July, 2001
%%
%% MAKE SURE THAT THE FILE HAS BEEN PERSONALIZED BEFORE YOU
%% PRINT AND SHIP THE FINAL VERSION.  YOU CAN FIND ITEMS THAT NEED
%% TO BE PERSONALIZED BY SEARCHING FOR THE STRING ``%PERSONALIZE''
%%
%%
%%first of all the cover.
{\pagestyle{empty}
\newcommand{\printtitle}{%
{\Huge\bf The ILLC Dissertation\\[0.8cm] Style}}    %PERSONALIZE

\begin{titlepage}
\par\vskip 2cm
\begin{center}
\printtitle
\vfill
{\LARGE\bf John B. Goode}                           %PERSONALIZE
\vskip 2cm
\end{center}
\end{titlepage}
%
% Skip a page to start on a right page again.
% If you're printing single-sided, simply delete    %PERSONALIZE
% the following line.
%
\mbox{}\newpage
\setcounter{page}{1}

%%the very first page: the `franse pagina'
\par\vskip 2cm
\begin{center}
\printtitle
\end{center}

%%the second page: the `illc pagina'
\clearpage
\par\vskip 2cm
\begin{center}
ILLC Dissertation Series DS-200X-NN                 %PERSONALIZE
\par\vspace {2cm}
\illclogo{10cm}
\par\vspace {2cm}
\noindent%
For further information about ILLC-publications, please contact\\[2ex]
Institute for Logic, Language and Computation\\
Universiteit van Amsterdam\\
Science Park 107\\
1098 XG Amsterdam\\
phone: +31-20-525 6051\\
e-mail: {\tt illc@uva.nl}\\
homepage: {\tt http://www.illc.uva.nl/}
\end{center}
\vfill

% If you're supported by NWO adapt the following 6 lines;
% otherwise simply delete them.
%
\noindent%
The investigations were supported by the            %PERSONALIZE
Philosophy Research Foundation\linebreak (SWON), 
which is subsidized by the Netherlands 
Organization for Scientific\linebreak Research (NWO).
\par\vspace {2cm}

% If you want to add CIP data (a summary of all the data used in
% library catalogs in a standard format), uncomment the following
% three lines and add the CIP data in between
%
%\noindent%
%{\tt CIP gegevens}                                 % PERSONALIZE
%\\[4ex]                                            %PERSONALIZE

%Copyright and accreditation stuff, plus ISBN
%
\noindent%
% Copyright: put your name here
Copyright \copyright\ 200X by John B.\ Goode\\[2ex] %PERSONALIZE
% Cover design, if your cover was designed by someone else
Cover design by Bert Jones.\\                       %PERSONALIZE
%Maybe some additional info on the production of the dissertation.
%Don't forget your printing shop
Printed and bound by your printer.\\[2ex]           %PERSONALIZE
%ISBN number: ask your faculty library how to obtain one
ISBN: 90--XXXX--XXX--X                              %PERSONALIZE

% Dedication, table of contents and acknowledgements
% are handled in the main file


%%the third page: the `titelblad'
\clearpage
\par\vskip 2cm
\begin{center}
\printtitle
\par\vspace {6cm}
{\large \sc Academisch Proefschrift}
\par\vspace {1cm}
{\large ter verkrijging van de graad van doctor aan de\\
Universiteit van Amsterdam\\
op gezag van de Rector Magnificus\\
prof.dr. D.C. van den Boom\\                                 %PERSONALIZE
ten overstaan van een door het college voor\\
promoties ingestelde commissie, in het openbaar\\
te verdedigen in de Aula der Universiteit \\        %PERSONALIZE
% Note: If your UvA PhD defense is located at the Agnietenkapel, simply write
% 'te verdedigen in de Agnietenkapel \\', i.e. do not add 'der Universiteit'
op maandag 1 januari 2001, te 12.00 uur \\ }        %PERSONALIZE
\par\vspace {1cm} {\large door}
\par \vspace {1cm} % Note: next should be your _full_ name
{\Large John Benedict Goode}                        %PERSONALIZE
\par\vspace {1cm} % and your birthplace
{\large geboren te Alice Springs, Verenigde Staten van Amerika.} %PERSONALIZE
% Note: include your country of birth IF AND ONLY IF you were not born
% in the Netherlands
\end{center}

%%the fourth page: promotores, ISBN etc.
\clearpage
\noindent%
\begin{tabular}[t]{@{}ll}
Promotor:      & Prof.dr.\ J.~Smith\\                %PERSONALIZE
Co-promotor:   & Dr.\ T.~Jones\\                     %PERSONALIZE
\\
Overige leden: & Prof. Dr. A. Aap\\                  %PERSONALIZE
               & Prof. Dr. B. Benson\\               %PERSONALIZE
               & Dr. C. Cornelissen\\                %PERSONALIZE
\end{tabular}\\
\\
Faculteit der Natuurwetenschappen, Wiskunde en Informatica\\ %PERSONALIZE
\clearpage
} % Back to \pagestyle{plain}

%%%%%%%%%%%%%%%%%%%%%%%END of FRONT MATTER%%%%%%%%%%%%%%%%%%%%%%%%%%%

%% BTH Dissertation Style 2
%% Blekinge Institute of Technology alternative dissertation/thesis template. See https://github.com/CaterHatterPillar/bth-dissertation-style for a style resembling the official, close-source, dissertation template commonly used in BTH computer science theses.
%%
%% Based on the ILLC Dissertation Style by the Institute for Logic, Language and Computation, University of Amsterdam (used with permission).
%% Licensed under the BSD 3-Clause License.
%% See adjoined LICENSE for more information.

\thispagestyle{plain}
\mbox{}
\vspace{2in}
\begin{center}
{\em to me \\ \ \\
who did all the work on this}
\footnote{The dedication is optional}
\end{center}

\tableofcontents
\acknowledgments

I am very grateful to Prof.dr.\ J.\ Smith whose help proved really extremely
invaluable.\\[2ex]				
Alice Springs\hfill John B. Goode\\
October, 200X.



%%  now we can start with the real thing

\cleardoublepage
\pagestyle{headings}
\pagenumbering{arabic}
  
    <your dissertation>

%%  \include the `end matter'

\begin{thebibliography}{XX}
\bibitem{Comment}
According to ILLC standards a chapter containing bibliographic
references should always be included in your dissertation.
It is specified by:
\begin{verbatim}
  \begin{thebibliography}{XX}
    <your list of \bibitems>
  \end{thebibliography}
\end{verbatim}
\bibitem{Lamport}
L. Lamport. {\em \LaTeX{} User's Guide \& Reference
Manual\/}, Addison-Wesley Publishing Company, Reading, Mass. 1986, 1994.
\end{thebibliography}



\begin{theindex}
By preference, your dissertation\linebreak
should contain an index. Instructions
on how to produce an index can be
found on pages 77--79 of the
 \LaTeX\ manual. You may specify
an index as follows:\\[2ex]
\verb|  \begin{theindex}|\\
\verb|    <your list of entries>|\\
\verb|  \end{theindex}|
\end{theindex}


%% BTH Dissertation Style 2
%% Blekinge Institute of Technology alternative dissertation/thesis template. See https://github.com/CaterHatterPillar/bth-dissertation-style for a style resembling the official, close-source, dissertation template commonly used in BTH computer science theses.
%%
%% Based on the ILLC Dissertation Style by the Institute for Logic, Language and Computation, University of Amsterdam (used with permission).
%% Licensed under the BSD 3-Clause License.
%% See adjoined LICENSE for more information.

\begin{thesymbols}
This is an optional chapter containing a list of symbols that
you use. It is specified by:\\[2ex]
\verb|  \begin{thesymbols}|\\
\verb|    <your list of symbols>|\\
\verb|  \end{thesymbols}|
\end{thesymbols}

\samenvatting
According to both ILLC standards and UvA promotion regulations,
a chapter containing a summary in
Dutch of your dissertation should always be included.
It is specified by:
\begin{verbatim}
  \samenvatting
    <your Samenvatting>
\end{verbatim}

%% BTH Dissertation Style 2
%% Blekinge Institute of Technology alternative dissertation/thesis template. See https://github.com/CaterHatterPillar/bth-dissertation-style for a style resembling the official, close-source, dissertation template commonly used in BTH computer science theses.
%%
%% Based on the ILLC Dissertation Style by the Institute for Logic, Language and Computation, University of Amsterdam (used with permission).
%% Licensed under the BSD 3-Clause License.
%% See adjoined LICENSE for more information.

\abstract
According to both ILLC standards and UvA promotion regulations,
an abstract of your dissertation in English should always be included.
This chapter may be specified by:
\begin{verbatim}
  \abstract
    <your Abstract>
\end{verbatim}


\curriculum
This is an optional chapter containing your Curriculum Vitae.
It is specified as follows:
\begin{verbatim}
  \curriculum
    <your CV>
\end{verbatim}



%%  finally, \include the list of previous ILLC dissertations

\pagestyle{empty}

\noindent
{\em Titles in the ILLC Dissertation Series:}

\newcommand{\illcpublication}[3]{\item[ILLC #1: ]{\bf #2}\\{\em #3}}

\begin{list}{}{ \settowidth{\leftmargin}{ILL}
		\setlength{\rightmargin}{0in}
		\setlength{\labelwidth}{\leftmargin}
		\setlength{\labelsep}{0in}
}

\illcpublication{DS-2009-01}{Jakub Szymanik}{Quantifiers in TIME and SPACE. Computational Complexity of Generalized Quantifiers in Natural Language}
\illcpublication{DS-2009-02}{Hartmut Fitz}{Neural Syntax}
\illcpublication{DS-2009-03}{Brian Thomas Semmes}{A Game for the Borel Functions}
\illcpublication{DS-2009-04}{Sara L. Uckelman}{Modalities in Medieval Logic}
\illcpublication{DS-2009-05}{Andreas Witzel}{Knowledge and Games: Theory and Implementation}
\illcpublication{DS-2009-06}{Chantal Bax}{Subjectivity after Wittgenstein. Wittgenstein's embodied and embedded subject and the debate about the death of man.}
\illcpublication{DS-2009-07}{Kata Balogh}{Theme with Variations. A Context-based Analysis of Focus}
\illcpublication{DS-2009-08}{Tomohiro Hoshi}{Epistemic Dynamics and Protocol Information}
\illcpublication{DS-2009-09}{Olivia Ladinig}{Temporal expectations and their violations}
\illcpublication{DS-2009-10}{Tikitu de Jager}{``Now that you mention it, I wonder\ldots'': Awareness, Attention, Assumption}
\illcpublication{DS-2009-11}{Michael Franke}{Signal to Act: Game Theory in Pragmatics}
\illcpublication{DS-2009-12}{Joel Uckelman}{More Than the Sum of Its Parts: Compact Preference Representation Over Combinatorial Domains}
\illcpublication{DS-2009-13}{Stefan Bold}{Cardinals as Ultrapowers. A Canonical Measure Analysis under the Axiom of Determinacy.}
\illcpublication{DS-2010-01}{Reut Tsarfaty}{Relational-Realizational Parsing}
\illcpublication{DS-2010-02}{Jonathan Zvesper}{Playing with Information}
\illcpublication{DS-2010-03}{C\'edric D\'egremont}{The Temporal Mind. Observations on the logic of belief change in interactive systems}
\illcpublication{DS-2010-04}{Daisuke Ikegami}{Games in Set Theory and Logic}
\illcpublication{DS-2010-05}{Jarmo Kontinen}{Coherence and Complexity in Fragments of Dependence Logic}
\illcpublication{DS-2010-06}{Yanjing Wang}{Epistemic Modelling and Protocol Dynamics}
\illcpublication{DS-2010-07}{Marc Staudacher}{Use theories of meaning between conventions and social norms}
\illcpublication{DS-2010-08}{Am\'elie Gheerbrant}{Fixed-Point Logics on Trees}
\illcpublication{DS-2010-09}{Ga\"elle Fontaine}{Modal Fixpoint Logic: Some Model Theoretic Questions}
\illcpublication{DS-2010-10}{Jacob Vosmaer}{Logic, Algebra and Topology. Investigations into canonical extensions, duality theory and point-free topology.}
\illcpublication{DS-2010-11}{Nina Gierasimczuk}{Knowing One's Limits. Logical Analysis of Inductive Inference}
\illcpublication{DS-2010-12}{Martin Mose Bentzen}{Stit, Iit, and Deontic Logic for Action Types}
\illcpublication{DS-2011-01}{Wouter M. Koolen}{Combining Strategies Efficiently: High-Quality Decisions from Conflicting Advice}
\illcpublication{DS-2011-02}{Fernando Raymundo Velazquez-Quesada}{Small steps in dynamics of information}
\illcpublication{DS-2011-03}{Marijn Koolen}{The Meaning of Structure: the Value of Link Evidence for Information Retrieval}
\illcpublication{DS-2011-04}{Junte Zhang}{System Evaluation of Archival Description and Access}
\illcpublication{DS-2011-05}{Lauri Keskinen}{Characterizing All Models in Infinite Cardinalities}
\illcpublication{DS-2011-06}{Rianne Kaptein}{Effective Focused Retrieval by Exploiting Query Context and Document Structure}
\illcpublication{DS-2011-07}{Jop Bri\"et}{Grothendieck Inequalities, Nonlocal Games and Optimization}
\illcpublication{DS-2011-08}{Stefan Minica}{Dynamic Logic of Questions}
\illcpublication{DS-2011-09}{Raul Andres Leal}{Modalities Through the Looking Glass: A study on coalgebraic modal logic and their applications}
\illcpublication{DS-2011-10}{Lena Kurzen}{Complexity in Interaction}
\illcpublication{DS-2011-11}{Gideon Borensztajn}{The neural basis of structure in language}
\illcpublication{DS-2012-01}{Federico Sangati}{Decomposing and Regenerating Syntactic Trees}
\illcpublication{DS-2012-02}{Markos Mylonakis}{Learning the Latent Structure of Translation}
\illcpublication{DS-2012-03}{Edgar Jos\'e Andrade Lotero}{Models of Language: Towards a practice-based account of information in natural language}
\illcpublication{DS-2012-04}{Yurii Khomskii}{Regularity Properties and Definability in the Real Number Continuum: idealized forcing, polarized partitions, Hausdorff gaps and mad families in the projective hierarchy.}
\illcpublication{DS-2012-05}{David Garc\'ia Soriano}{Query-Efficient Computation in Property Testing and Learning Theory}
\illcpublication{DS-2012-06}{Dimitris Gakis}{Contextual Metaphilosophy - The Case of Wittgenstein}
\illcpublication{DS-2012-07}{Pietro Galliani}{The Dynamics of Imperfect Information}
\illcpublication{DS-2012-08}{Umberto Grandi}{Binary Aggregation with Integrity Constraints}
\illcpublication{DS-2012-09}{Wesley Halcrow Holliday}{Knowing What Follows: Epistemic Closure and Epistemic Logic}
\illcpublication{DS-2012-10}{Jeremy Meyers}{Locations, Bodies, and Sets: A model theoretic investigation into nominalistic mereologies}
\illcpublication{DS-2012-11}{Floor Sietsma}{Logics of Communication and Knowledge}
\illcpublication{DS-2012-12}{Joris Dormans}{Engineering emergence: applied theory for game design}
\illcpublication{DS-2013-01}{Simon Pauw}{Size Matters: Grounding Quantifiers in Spatial Perception}
\illcpublication{DS-2013-02}{Virginie Fiutek}{Playing with Knowledge and Belief}
\illcpublication{DS-2013-03}{Giannicola Scarpa}{Quantum entanglement in non-local games, graph parameters and zero-error information theory}
\illcpublication{DS-2014-01}{Machiel Keestra}{Sculpting the Space of Actions. Explaining Human Action by Integrating Intentions and Mechanisms}
\illcpublication{DS-2014-02}{Thomas Icard}{The Algorithmic Mind: A Study of Inference in Action}
\illcpublication{DS-2014-03}{Harald A. Bastiaanse}{Very, Many, Small, Penguins}

\end{list}



\end{document}
\end{verbatim}
%
%
If your file is already coded with \LaTeX{} you can easily
adapt it a posteriori to the ILLC Dissertation Style.

If your document is coded with the ILLC Dissertation Style,
you may not be able to typeset it using the standard \LaTeX\ book style
without doing some minor recoding,
as the ILLC Dissertation Style file defines some comands 
that are not provided by the standard \LaTeX\ book style.

Please refrain from using any \LaTeX{} or \TeX{} commands
that affect the layout or formatting of your document
(i.e.\ commands like \verb|\textheight|, \verb|\hoffset| etc.).
The ILLC Dissertation Style has been carefully designed 
to produce the rightlayout from your \LaTeX\ input.
There may nevertheless be exceptional occasions on which to use some of them.
If there is anything specific you would like to do 
and for which neither \LaTeX{} nor the ILLC Dissertation Style file 
provides a command,
{\em please contact us\/} (email: illc@science.uva.nl).

\section{Personalizing {\tt guide\_front.tex}}
The file {\tt guide\_front.tex} contains all information needed 
to produce the front matter of your dissertation according to ILLC standards.
You need to personalize {\tt guide\_front.tex} by inserting your data
at appropriate spots. Additionally, those that print on single-sided
printers will want to eliminate the empty page printed after the cover page.
All items that need to be personalized in 
{\tt guide\_front.tex} can be found by searching for the string 
``\%{}PERSONALIZE''.
Note that items such as the name of the Rector Magnificus {\em may} be 
up-to-date, but it is prudent to assume not and research the name of
the current Rector Magnificus. All such items are also marked with
the string ``\%{}PERSONALIZE''.
The files {\tt guide\_dedication} and {\tt guide\_acknowledgements} 
contain the optional dedication and acknowledgements, and are \verb|\include|d
from the main file.
You can personalize the text in these files, or simply change the name
of the \verb|\include|d files in the main file.

\section{Personalizing {\tt guide\_spine.tex}}
The file {\tt guide\_spine.tex} contains all information needed
to produce the spine of your dissertation according to ILLC standards.
You need to personalize the file {\tt guide\_spine.tex} by inserting 
your data at appropriate spots. All items that need to be personalized in
{\tt guide\_spine.tex} can be found by searching for the string
``\%{}PERSONALIZE''.
