%% BTH Dissertation Style 2
%% Blekinge Institute of Technology alternative dissertation/thesis template. See https://github.com/CaterHatterPillar/bth-dissertation-style for a style resembling the official, close-source, dissertation template commonly used in BTH computer science theses.
%%
%% Based on the ILLC Dissertation Style by the Institute for Logic, Language and Computation, University of Amsterdam (used with permission).
%% Licensed under the BSD 3-Clause License. See adjoined LICENSE for more information.
%% Currently maintained by Eric Nilsson.
%%
%% Version: June 29, 2014.
%%

\chapter{Dissertation Layout}
The following describes how your dissertation may be organized.
Note that other restrictions may apply if using the template at Blekinge Institute of Technology.

\section{The cover}
The BTH Dissertation Style only prescribes the font, size and location 
of the title and author on the cover page. Besides this you are free to 
design your own cover.

If applicable, dissertations may have a spine displaying the authors name, the title of the dissertation, and the BTH logo.
There is a file called {\tt bthdiss\_spine.tex} to help you format your spine text.

\section{The front matter}
The front matter has Roman page numbers (this is achieved by specifying the command \verb|\pagenumbering{roman}| after the \verb|\begin{document}| declaration).
The front matter may contain the following material in the following order:

\begin{enumerate}
  \item[i] ``franse pagina'' containing nothing but the title of your dissertation.
  \item[ii] the ``BTH page'' containing the logo and address of BTH.
  \item[iii] the title page containing the text prescribed by the university.
  \item[iv] this page contains the following information in the following order:
    \begin{itemize}
      \item name and address of your promotor.
      \item when appropriate, an acknowledgment to collaborating organizations.
      \item copyright notices (if applicable).
      \item information concerning the production of your dissertation
      \item ISBN (if applicable)
	\end{itemize}
  \item[v] (optional) dedication
  \item[v] (or vii) table of contents
  \item[vii] (or ix) Acknowledgments, specified by \verb|\acknowledgments|.
\end{enumerate}

The file called  \verb|bthdiss_front.tex| helps you format the front matter of your dissertation.

\section{The body of your text}
This section contains some information about organizing the main
text of your dissertation.

\paragraph*{Headings.}
Headings will be automatically generated by the following codes:
\begin{verbatim}
  \chapter
  \section
  \subsection
  \subsubsection
  \paragraph
\end{verbatim}
The headings produced by \verb|\paragraph| and \verb|\subparagraph| need to be punctuated at the end, as they are followed by the body of the (sub-)paragraph.

\paragraph*{Theorem-like environments.}
In addition to the above headings your text may be structured by theorem-like environments, like lemmas, propositions, conjectures, \ldots .
The following theorem-like environments are predefined by the BTH Disseration Style: \verb|theorem|, \verb|lemma|, \verb|corollary|, \verb|conjecture|, \verb|proposition|, \verb|definition|, \verb|remark|, \verb|example|, \verb|convention|, \verb|fact| and \verb|question|.
They are defined to be numbered consecutively, i.e. typing:
\begin{verbatim}
\begin{lemma}
  This is a lemma
\end{lemma}
\begin{question}
  Is this a question?
\end{question}
\end{verbatim}
produces:
\begin{lemma}
  This is a lemma
\end{lemma}
\begin{question}
  Is this a question?
\end{question}

A number of theorem-like environments have italicized text: \verb|theorem|, \verb|lemma|, \verb|corollary|, \verb|conjecture| and \verb|proposition|.
All other pre-defined environments have roman text.
Inside theorem-like environments text may be emphasized by using \verb|\em|. (In environments with italicized text such as lemma and theorems this will produce text in roman type style; in environments with roman text this produces italicized text.)
As a rule of thumb you should always emphasize the terms being defined in a definition.

\paragraph*{Special signs and characters.}
You may need to use special signs. The available ones are listed
in the \LaTeX{} {\em User's Guide \& Reference Manual\/}, pp.~44 ff.

\paragraph*{Splitting your input}
Rather than putting the whole input of a document in a single file, you may wish to split it into several smaller ones.
There will always be one file that is the {\em root} file; it is the one whose name you type when you run \LaTeX{}.
The root file of the document you are reading is called \verb|bthdiss.tex|.
Other files may be `included' by the commands \verb|\input| and \verb|\include|.
The command \verb|\input{filename}| causes \LaTeX{} to insert the contents of the file \verb|filename.tex| right at the current spot in your manuscript.
The command \verb|\include{filename}| does the same, except that the included text will begin and end on its own page (i.e. an automatic \verb|\clearpage| command is issued at the beginning and end of the included file).
Additionally, this allows the use of the \verb|\includeonly| command (see the paragraph on saving paper).
The \verb|\include| command is the preferred way to include a file containing, for instance, the text of a single chapter.

\section{The end matter}
The end matter should at least contain a Bibliography and an Abstract.
Alternatively your dissertation also contains an Index.
In addition it may contain Appendices, a List of Symbols and your Curriculum Vitae.
\begin{itemize}
\item
Appendices (optional), see pp.\ 23, 158 of the
  \LaTeX{} {\em User's Guide \& Reference Manual\/} on how to create
  appendices 
\item
Bibliography (obligatory), specified by 
\begin{verbatim}
  \begin{thebibliography}{XX}
    <your list of \bibitems>
  \end{thebibliography}
\end{verbatim}
\item
Index, specified by
\begin{verbatim}
  \begin{theindex}
    <your list of entries>
  \end{theindex}
\end{verbatim}
\item
List of Symbols (optional), specified by
\begin{verbatim}
  \begin{thesymbols}
    <your list of symbols>
  \end{thesymbols}
\end{verbatim}
\item
Abstract (obligatory), specified by
\begin{verbatim}
  \abstract
    <your Abstract>
\end{verbatim}
\item
Curriculum Vitae (optional), specified by
\begin{verbatim}
  \curriculum
    <your CV>
\end{verbatim}
\end{itemize}
The end matter of this document has been split into separate files, which \verb|included| in the main file.

\section{The spine}
You can use the file {\tt bthdiss\_spine.tex} to typeset the text on the spine of your dissertation.
This text should consist of your name, the title of your dissertation, and the BTH logo.

The file {\tt bthdiss\_spine.tex} produces the text for the spine of your dissertation in a number of sizes.
Let your competent printer choose the most appropriate size.
